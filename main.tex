\documentclass[twoside]{article}
\usepackage{graphicx}
\usepackage[top=2.5cm,left=2.5cm,right=2.5cm,bottom=2.5cm,headsep=0.3in,headheight=1in]{geometry}
\usepackage{multirow}
\usepackage[ngerman]{babel}
\usepackage{float}
\usepackage{fancyhdr}


%Eingabe der Parameter in die letzten geschwungenen Klammern

\newcommand{\titeltitel}{}                  %Titel der Übung
\newcommand{\titelraumbezeichnung}{}        %Raumbezeichnung
\newcommand{\titelgruppe}{}                 %Gruppe
\newcommand{\titellehrer}{}                 %Lehrer
\newcommand{\titeluebungsnummer}{}          %Übungsnummer
\newcommand{\titelabgabedatum}{}            %Abgabedatum
\newcommand{\titelkatalognummer}{}          %Katalognummer
\newcommand{\titeluebungdatum}{}            %Übungsdatum
\newcommand{\titelnameprotokoll}{}          %Name des Protokollist
\newcommand{\titelnameteam}{}               %Name der Teammitglieder
\newcommand{\titelklasse}{}                 %Klasse
\newcommand{\titelgeprueftdatum}{}          %Prüfungsdatum  %Definiert Variablen für das Titelblatt

\begin{document}

%Titelblatt
\thispagestyle{empty}
\newgeometry{top=20mm, left=18.25mm, right=18.25mm, bottom=20mm}
\noindent
\begin{tabular}{|p{4.5cm}|p{7.5cm}|p{2.5cm}|p{2cm}|} %17.35mm
    \hline
    \vspace{0.4cm}
    \multirow{2}{4.5cm}{\hspace{0.7cm}\includegraphics{img/titel/Labor_Protokoll_Logo.jpg}} & \multirow{2}{7.5cm}{\begin{center}{\huge Laboratorium}\\\vspace{0.25cm} Raumbezeichnung: \hspace{0.25cm} \titelraumbezeichnung \end{center}} & \vspace{0.05cm} Katalognummer:\vspace{0.3cm} & \vspace{0.06cm}\hspace{0.7cm}{\Large\titelkatalognummer}\\
    \cline{3-4}
     & & \vspace{0.05cm}Tag der Übung:\vspace{0.3cm} & \vspace{0.14cm} {\large\titeluebungdatum} \\
    \hline
    \vspace{0.35cm} {\large Gruppe: \titelgruppe} & \vspace{0.1cm} {Protokoll: \hspace{0.27cm} \titelnameprotokoll \vspace{0.1cm} \newline Mitglieder: \hspace{0.1cm} \titelnameteam} \vspace{0.2cm} & \vspace{0.35cm} {\large Klasse} & \vspace{0.35cm} {\Large\titelklasse} \\
    \hline
    \multicolumn{4}{|p{17cm}|}{%Bei Bild vspace unten ändern
    \vspace{18cm}}\\
    \hline
\end{tabular}

\vspace{-1px}
\noindent
\begin{tabular}{|p{2.3cm}|p{1.8cm}|p{7.9cm}|p{2.5cm}|p{1.58cm}|}
    \vspace{0.1cm}Lehrer\vspace{0.2cm} & \vspace{0.1cm}\titellehrer & \vspace{0.15cm} \multirow{2}{7.1cm}{\centerline{Titel der Übung}\vspace{0.2cm}\newline\centerline{\huge \textbf{\titeltitel}}} & \vspace{0px} Übungsnummer & \vspace{1px}\titeluebungsnummer\\
    \cline{1-2}\cline{4-5}
    \vspace{0px}Geprüft\vspace{0.15cm} & \vspace{0px} \titelgeprueftdatum &  & \vspace{0px}Abgabe am\vspace{0.15cm} & \vspace{0px}\titelabgabedatum \\
    \hline
\end{tabular}
\newgeometry{top=2.5cm,left=2.5cm,right=2.5cm,bottom=2.5cm,headsep=0.3in,headheight=1in}

%Kopf- und Fußzeile
\pagestyle{fancy}
\fancyhead[R]{\titelnameprotokoll} %Protokollist oben rechts
\fancyhead[L]{\titeluebungdatum}   %Übungsdatum oben links
\fancyfoot[C]{\titeltitel}         %Titel der Übung unten mittig
\fancyfoot[R]{\thepage}           %Seitennummer unten außen

%Inhaltsverzeichnis
\tableofcontents
\newpage

%Begin der Dokumentation
\section{}

\end{document}
